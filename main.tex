\documentclass[a11paper, 11pt]{article}

\usepackage{document}
\usepackage{titlepage}
\usepackage[T1]{fontenc}
\usepackage[french]{babel}

% Optional packages and tools (Uncomment to your needs)
\usepackage{tools/bibliography}
\usepackage{tools/toolbox}
\usepackage{tools/engineering}
\usepackage{tools/french}

% To add a bibliography, first make sure that your file
% is named "bibliography.bib" and then uncomment the
% following line.

%\addbibresource{bibliography.bib}


% Uncomment to compile links and references...
%\nofiles

\institution{Université de Sherbrooke}
\faculty{Faculté de...}
\department{Département de...}
\title{Titre du document}
\class{Nom du cour}
\classnb{ABC123}
\author{
  \addtolength{\tabcolsep}{-0.4em}
  \begin{tabular}{rcl} % Ajouter des auteurs au besoin
  Prénom Nom     & -- & CIP \\
  Prénom2 Nom    & - & CIP \\
  \end{tabular}
}
\teacher{John Doe}
\location{Sherbrooke}
\date{\today}

\begin{document}
\maketitle
\newpage
\tableofcontents
\newpage

\section{This is the first section}
This is some text...

\subsection{This is a subsection}
This is some more text!


\section{Here are Some Exemples}

\subsection{Making a Table}

Tables are a scary subject to start with in latex.
This is mostly because their systax seems strange to an untrained eye.
Fortunately there are tools to make this step easier.
Consider visiting \href{https://www.tablesgenerator.com/} {tables generator}.


Here is an exemple of a complex table directly copied from \href{https://www.tablesgenerator.com/} {tables generator}.
\\
Note how the site advices the use of some packages in the commented section.
Good practice would want these packages added in the \textit{document.sty} file but they can also be added at the begining
of your \textit{main.tex} file. Adding these lines at a wrong location may result in a warning or your
document failing to compile.


% Please add the following required packages to your document preamble:
% \usepackage[table,xcdraw]{xcolor}
% Beamer presentation requires \usepackage{colortbl} instead of \usepackage[table,xcdraw]{xcolor}
% \usepackage[normalem]{ulem}
% \useunder{\uline}{\ul}{}
\begin{table}[h!]
\begin{tabular}{ccc}
\multicolumn{3}{c}{A Not So Randomly Generated Title} \\ \hline
\multicolumn{1}{c|}{\textbf{Yeah}} &
  \multicolumn{1}{c|}{\textbf{Yah}} &
  \textbf{Yea} \\ \hline
\multicolumn{1}{c|}{{\color[HTML]{656565} \textit{Lawful Good}}} &
  \multicolumn{1}{c|}{{\color[HTML]{656565} \textit{Neutral Good}}} &
  {\color[HTML]{656565} \textit{Chaotic Good}} \\ \hline
\multicolumn{1}{c|}{\textbf{Yes}} &
  \multicolumn{1}{c|}{\textbf{Ya}} &
  \textbf{Yeet} \\ \hline
\multicolumn{1}{c|}{{\color[HTML]{656565} \textit{Lawful Neutral}}} &
  \multicolumn{1}{c|}{{\color[HTML]{656565} \textit{True Neutral}}} &
  {\color[HTML]{656565} \textit{Chaotic Neutral}} \\ \hline
\multicolumn{1}{c|}{\textbf{Yep}} &
  \multicolumn{1}{c|}{\textbf{Ye}} &
  \textbf{Yuh} \\ \hline
\multicolumn{1}{c|}{{\color[HTML]{656565} \textit{Lawful Evil}}} &
  \multicolumn{1}{c|}{{\color[HTML]{656565} \textit{Neutral Evil}}} &
  {\color[HTML]{656565} \textit{Chaotic Evil}}
\end{tabular}
\end{table}


% How to insert an image:

% \begin{figure}[t!]
%     \centering
%     \includegraphics[width=0.45\textwidth]{/path/to/image.jpg}
%     \caption{Caption Text}
%     \label{fig: A Relateble Name}
% \end{figure}


% \newpage
% \printbibliography[heading=bibintoc]
\end{document}
